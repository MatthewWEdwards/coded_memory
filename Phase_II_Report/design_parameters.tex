In this section, we discuss various design parameters that we use in this phase of the project to design and simulate the efficient code storage. \\
\textit{Memory overhead}: The gains of multiple accesses to a memory bank every cycle comes with an associated cost. The cost is to store the compressed redundancy or the codes. The extra memory space used to store these codes is limited to 15$\%$ of the overall memory.\\
\textit{Memory size}: is an important parameter for consideration in design. The memory size and the parity storage size decide the code function to be used to essentially compress the redundant data. This design is considered for memory size of 8kB � 256 kB. This is the shared part of the memory which would be accessed by all the cores. \\ 
\textit{Memory Banks}: The memory banks essentially are the units which store the data. We consider the code design for 8 memory banks. We consider the memory banks addressed with Least Significant Bits (LSBs) of the address. The last 3 bits of the address decide which bank, the memory address belongs to and the rest of the MSBs decide the location within the bank.\\ 
\textit{Cache line size}: The memory accesses happen in burst as a cache line is evicted from the cache and is requested to be replaced by the cores. The cache controller thus requests a cache line which is a starting address and the length of the cache line. In this design, we consider cache line size of 128 bytes and 256 bytes. However, each core can potentially have a different cache line size and the concept of coding could be extended for various sizes.\\ 
\textit{Element size}:  Each memory location in a memory bank stores 256 bit of data. This essentially relates to decoding/understanding the address access request to the memory bank. The cores request memories to be read or written for multiple elements. For example, a core with 128 bytes of cache line would request 4 elements of read/write for each cache line. The shared traces have two different request patterns, for 128 bytes and for 256 bytes. \\
\textit{Number of Cores}: This parameter refers number of cores making access to the memory controller. This parameter is not used in the design of the coding scheme. However, we validate the design using the 6 core access trace shared with us for LTE and UMTS. \\
\textit{Access rate}: This is the average rate at which the memory controller executes the reads/writes to the memory banks. In this design, we consider 1.54 ns as the access rate. This would mean that the clock rate to memory would be at 650MHz. This parameter is required to simulate the performance for the shared traces. 
