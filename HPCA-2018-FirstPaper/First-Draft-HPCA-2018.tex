%%%%%%%%%%%%%%%%%%%%%%%%%%%%%%%%%%%%
% This is the template for submission to HPCA 2018
% The cls file is a modified from  'sig-alternate.cls'
%%%%%%%%%%%%%%%%%%%%%%%%%%%%%%%%%%%%

\documentclass{sig-alternate}
\setlength{\paperheight}{11in}
\setlength{\paperwidth}{8.5in}

\newcommand{\ignore}[1]{}
\usepackage[pass]{geometry}
\usepackage{fancyhdr}
\usepackage[normalem]{ulem}
\usepackage[hyphens]{url}
\usepackage{hyperref}
\usepackage{array}
\usepackage{color}
\usepackage{enumitem}
\usepackage{caption}
\usepackage{subcaption}
\usepackage{subfloat}
\usepackage{graphics}
\usepackage{amsmath}
\usepackage{comment}

\newtheorem{theorem}{Theorem}[section]
\newtheorem{proposition}{Proposition}[section]
\newtheorem{lemma}{Lemma}[section]
\newtheorem{corollary}{Corollary}[section]
\newtheorem{claim}{Claim}[section]
\newtheorem{definition}{Definition}

%\newtheorem{definition}{Definition}%[section]
\newtheorem{example}{Example}%[section]
\newtheorem{conjecture}{Conjecture}%[section]
\newtheorem{construction}{Construction}%[section]
\newtheorem{remark}{Remark}%[section]

%\newtheorem{remark}{Remark}%[section]
\newtheorem{case}{Case}%[section]
\newtheorem{assumption}{Assumption}

%%%%%%%%%%%---SETME-----%%%%%%%%%%%%%
\newcommand{\hpcasubmissionnumber}{XXX}
\newcommand*\cn{{\color{red}{\textbf{[Citation Needed]}}}}
\newcommand*\Ankit[1]{{\color{red}{\textbf{[ANKIT:~#1]}}}}
\newcommand*\Matt[1]{{\color{red}{\textbf{[MATT:~#1]}}}}
%%%%%%%%%%%%%%%%%%%%%%%%%%%%%%%%%%%%

\fancypagestyle{firstpage}{
  \fancyhf{}
\setlength{\headheight}{50pt}
\renewcommand{\headrulewidth}{0pt}
  \fancyhead[C]{\normalsize{HPCA 2018 Submission
      \textbf{\#\hpcasubmissionnumber} -- Confidential Draft -- Do NOT Distribute!!}}
  \pagenumbering{arabic}
}

\setlist[itemize]{leftmargin=0.15in}
\setlist[enumerate]{leftmargin=0.15in}

%%%%%%%%%%%---SETME-----%%%%%%%%%%%%%
%\title{Dynamic Coding for Improved performance of Memories }
\title{Achieving Multi-Port Memory Performance on Single-Port Memory with Coding Techniques}
%%%%%%%%%%%%%%%%%%%%%%%%%%%%%%%%%%%%

\begin{document}
\maketitle
\thispagestyle{firstpage}
\pagestyle{plain}

%%%%%% -- PAPER CONTENT STARTS-- %%%%%%%%

\begin{abstract}
Many performance critical systems today must rely on performance enhancements, such as multi-port memories, to keep up with the increasing demand of memory-access capacity. However, the large area footprints and complexity of existing multi-port memory designs limit their applicability in practice. This paper explores a coding theoretic framework to address this problem. In particular, this paper introduces a framework to encode data across multiple single-port memory banks in order to {\em algorithmically} realize the functionality of multi-port memory.

This paper proposes three code designs with significant less storage overhead as compared to the existing replication based emulations of multi-port memories. To achieve optimal performance, we also demonstrate a memory controller design that utilizes redundancy across coded memory banks to most efficiently schedule read and write requests sent across multiple cores. Furthermore, guided by real-life traces, the paper explores two potential directions to improve the efficiency of the coding based memory design: 1) {\em Dynamic coding}, and 2) {\em Prefetching}. We then show significant performance improvements in critical word read and write latency in the proposed coded-memory design when compared to a traditional uncoded-memory design.
%%%%%%%%%%%%%%%%%%%%%%%%%%%%%%%%%%%
% Old abstract (ver2)
%%%%%%%%%%%%%%%%%%%%%
\begin{comment}
Many performance focussed systems need to rely on enhancements like multi-port memories to keep up with the increasing demand of access capacity, especially in a multi-core setup. However, the large area footprints and complexity of the existing designs of multi-port memories limits their applicability in practice. This paper explores a coding theoretic framework to address this problem in multi-port memory designs. 
In particular, this paper encodes the information across multiple single-port memory banks in order to {\em algorithmically} realize the functionality of a multi-port memory. 

This paper focuses on three specific code designs which have significantly less storage overhead as compared to the existing replication based emulations of multi-port memories. The paper also proposes novel memory controller designs that can utilize the redundancy among the memory banks to schedule/arbitrate the both read and write requests originating from different cores accessing the memory. Furthermore, guided by real-life traces, the paper explores two potential directions to improve the efficiency of the coding based memory design: 1) {\em Dynamic coding} and 2) {\em Prefetching}. This paper then performs extensive simulations to demonstrate the performance improvements attained by the proposed memory designs as compared to uncoded memory designs. These results show significant improvement in the critical word read and write latency with coded memory.
\end{comment}
%%%%%%%%%%%%%%%%%%%%%
% End of old abstract (ver2)
%%%%%%%%%%%%%%%%%%%%%%%%%%%%%%%%%%


%%%%%%%%%%%%%%%%%%%%%%%%%%%%%%%%%%%
% Old abstract
%%%%%%%%%%%%%%%%%%%%%
\begin{comment}
Designing memory units to keep up with the access requests from multiple cores is a major challenge for computer architects, {\color{red} especially in the face of growing computer sizes, increasing level of integration, and evident heterogeneity among various system components.} This forces many performance focussed systems to rely on enhancements like multi-port memories to meet the increasing demand of access capacity. However, the existing designs of multi-port memories incur large costs in terms of area, complexity, and {\color{red} need to redesign other components of the system.}

This paper explores a coding theoretic framework to address this problem in multi-port memory designs. %and enable retrieval mechanisms for fast information access. %This paper aims to enable retrieval mechanisms for fast information access while efficiently utilizing the storage space. 
%This is made possible by carefully designing low complexity encoding/decoding schemes to store information in memory banks in a redundant manner. These schemes are motivated by the coding techniques that are recently developed in the context of cloud storage systems. 
%Memory systems work hard to keep up with access requests from cores. Growing computer sizes, heterogenous systems and increasing level of integration has increased more. Performance focussed systems use enhancements like multi-port memories to increase the access capacity. However, they come with a cost in terms of area, complexity and cost of redesign and rebuilding a system. In this paper, we explore a mathematical solution to the problem where we explore an efficient memory storage and reterival mechanism for efficent access. 
In particular, this paper employs {\em codes with availability}, i.e., the codes with multiple disjoint ways to access a particular information block, to encode the information across memory banks to {\em algorithmically} realize the functionality of a multi-port memory. These designs have significantly less storage requirements as compared to existing replication based design. Specifically, the paper focuses on two designs corresponding to different underlying codes. The proposed memory designs are also accompanied with memory controllers that can utilize the redundancy among stored data to schedule/arbitrate the both read and write requests originating from different cores accessing the memory.  

Guided by real-life traces {\color{red}appearing in base station operations in cellular networks}, the paper then explores two potential directions to improve the efficiency of the coding based memory design in specific applications: 1) {\em Dynamic coding} and 2) {\em Prefetching}. %In particular, guided by some real-life traces {\color{red}{\bf [CONFIRM THIS?]} appearing in the context of base station operations in a cellular network.} 
These traces exhibit access patterns that are almost concentrated on small memory regions during different time windows. The dynamic coding approach exploits this by encoding only small portions of the information based on continuous detection of such regions in order to enable further savings of utilized storage space. Next, the prefetching based on pending access requests at memory controller is  explored to create the opportunities to serve as many requests as possible in a given time slot. %The design of such prefacing schemes crucially depends on the underlying coding scheme. 
This paper then analyze the performance improvements attained by the proposed memory designs as compared to uncoded memory designs. These results show significant improvement in critical word read and write latency with coded memory. %{\color{blue} Furthermore, the paper also provides intuitions derived from this analysis which can help the system designers increase the efficiency of coding based memory implementations.}
\end{comment}
%%%%%%%%%%%%%%%%%%%%%
% End of old abstract
%%%%%%%%%%%%%%%%%%%%%%%%%%%%%%%%%%
\end{abstract}

%%%%%%%%%%%%%%%%%%%%%%%%%%%%%%%%%%%%%%%%%%%%%%%%%%%%%%%%
% Introduction
%%%%%%%%%%%%%%%%%%%%%%%%%%%%%%%%%%%%%%%%%%%%
\section{Introduction}
\label{sec:intro}
Loading and storing information to memory is an intrinsic part of any computer program. As illustrated in Figure~\ref{fig:cpuvsmemory}, the past few decades have seen the performance gap between processors and memory grow. Even with the saturation and demise of Moore's law~\cite{Wulf1995, waldrop2016, MooreMITR}, processing power is expected to grow as multi-core architectures become more reliable~\cite{Geer}. The end-to-end performance of a program heavily depends on both processor and memory performance. Slower memory systems can bottleneck computational performance. This has motivated computer architects and researchers to explore strategies for shortening memory access latency, including sustained efforts towards enhancing the memory hierarchy~\cite{Burger}. Despite these efforts, long-latency memory accesses do occur when there is a miss in the last level cache (LLC). This triggers an access to shared memory, and the processor is stalled as it waits for the shared memory to return the requested information.
%---------------------------
\begin{figure}[t!]
\centering
\includegraphics[width=0.7\linewidth]{fig/cpuvsmemory.jpg}
\caption{\it{The gap in performance, measured as the difference in the time 
between processor memory requests (for a single processor or core) and the 
latency of a DRAM access, is plotted over a $30$ year span~\cite{comparchbook}.}}
\label{fig:cpuvsmemory}
\end{figure}
%---------------------------
In multi-core systems, shared memory access conflicts between cores result in large access request queues. Figure~\ref{fig:multicore_arch}  illustrates a general multi-core architecture. The bank queues are served every memory clock cycle and the acknowledgement with data is sent back to the corresponding processor. In scenarios where multiple cores request access to memory locations in the same bank, the memory controller arbitrates them using bank queues. This contention between cores to access from the same bank is known as a {\em bank conflict}. As the number of bank conflicts increases, the resultant increases in memory access latency causes the multi-core system to slow.

%---------------------------
\begin{figure}[t!]
\centering
\includegraphics[width=\linewidth]{fig/fig-2-memory-controller.png}
\caption{\it{General multi-core architecture with a shared memory. $N$ processor cores share a memory consisting of $M$ banks.}}
\label{fig:multicore_arch}
\end{figure}
%---------------------------
We address the issue of increased latency by introducing a coded memory design. The main principle behind our memory design is to distribute accesses intended for a particular bank across multiple banks. We redundantly store encoded data, and we decode memory for highly requested memory banks using idle memory banks. This approach allows us to simultaneously serve multiple read requests intended for a particular bank. Figure~\ref{fig:example_xor} shows this with an example. Here, Bank 3 is redundant as its content is a function of the content stored on Banks 1 and 2. Such redundant banks are also referred to as {\em parity banks}. Assume that the information is arranged in $L$ rows in two first two banks, represented by $[a(1),\ldots, a(L)]$ and $[b(1),\ldots, b(L)]$, respectively. Let $+$ denote the XOR operation, and additionally assume that the memory controller is capable of performing simple decoding operations, \textit{i.e.} recovering $a(j)$ from $b(j)$ and $a(j) + b(j)$. Because the third bank stores $L$ rows containing $[a(1) + b(1),\ldots, a(L) + b(L)]$, this design allows us to simultaneously serve any two read requests in a single memory clock cycle.   

%---------------------------
\begin{figure}[t!]
\centering
\includegraphics[width=0.395\linewidth]{fig/example-xor.pdf}
\caption{\it{Here the redundant memory in Bank 3 enables multiple read accesses to Bank 1 or 2. Given two read requests $\{a(i), a(j)\}$ directed to Bank $1$, we can deal with bank conflict in the following manner: 1) The first request for $a(i)$ is directly served by Bank $1$ itself.  2) The read request for $a(j)$ is served by downloading $b(j)$ and $a(j) + b(j)$ from Bank 2 and Bank 3, respectively. Another case where two read requests corresponding to two different banks, e.g., $\{a(i), b(j)\}$, can be simultaneously served from their respective banks without utilizing Bank $3$.}}
\label{fig:example_xor}
\end{figure}
%---------------------------
Hybrid memory designs such as the one in Figure~\ref{fig:example_xor} have additional requirements in addition to serving read requests. The presence of redundant parity banks raises a number of challenges while serving write requests. The memory overhead of redundant memory storage adds to the overall cost of such systems, so efforts must be made to minimize this overhead. Finally, the heavy memory access request rate possible in multi-core scenarios necessitates sophisticated scheduling strategies to be performed by the memory controller. In this paper we address these design challenges and evaluate potential solutions in a simulated memory environment. 

\noindent \textbf{Main contributions and organization:~}In this paper we systematically address all key issues pertaining to a shared memory system that can simultaneously service multiple access requests in a multi-core setup. We present all the necessary background on realization of multi-port memories using single-port memory banks along with an account of relevant prior work in Section~\ref{sec:bg}. We then present the main contributions of the paper which we summarize below. %Here, we highlight the main contributions of the paper.
\begin{itemize}
\item We focus on the design of the storage space in Section~\ref{sec:code_design}. In particular, we employ three specific coding schemes to redundantly store the information in memory banks. These coding schemes, which are based on the literature on distributed storage systems~\cite{dimakis, Gopalan12, batchcodes, RPDV16}, allow us to realize the functionality of multi-port memories from single port memories while efficiently utilizing the storage space. Moreover, these coding schemes have low complexity encoding and decoding processes that require only simple XOR operations. %We focus on two specific memory designs that store information in memory banks based on two different coding schemes from the literature on distributed storage systems (a.k.a. cloud storage systems)~\cite{dimakis, Gopalan12, batchcodes, RPDV16}. These coding schemes allow us to realize the functionality of multi-port memories from a single port memories while efficiently utilizing the storage space. Moreover, these coding schemes have low complexity encoding and decoding processes that require only simple XOR operation.
\item We present a memory controller architecture for the proposed coding based memory system in Section~\ref{sec:memcontrol}. Among other issues, the memory controller design involves devising scheduling schemes for both read and write requests. This includes careful utilization of the redundancy present in the memory banks while maintaining the validity of information stored in them.
%In our setup, these scheduling schemes need to take the underlying coding scheme into account in order to utilize the redundancy present in the array of memory banks in the best possible manner. Furthermore, we also address the issue of keeping track of the validity of the information stored in various banks. Note that, due to unserved previous write requests, some of the stored data might have become outdate as far as a particular read request is concerned.
%able to serve the masecond main component of a shared memory system, i.e., memory controller, in Section~\ref{sec:memcontrol}. The memory controller design We also design the memory controllers for the proposed memory systems based on the storage pattern in different memory banks. Note that the memory controller design involves devising buffering and arbitration (scheduling) schemes for both read and write requests.
\item Focusing on applications where memory traces might exhibit favorable access patterns, we explore two ways to improve the efficiency of our coding based memory design in Sections~\ref{sec:dynamicCoding} and~\ref{sec:prefetching}. First, we propose a dynamic coding scheme which is based on detection of continuous heavily accessed regions of memory. 
%The dynamic coding scheme only encode these heavily access regions at a particular time instance. As different (uncoded) regions begin receiving more accesses, the dynamic coding scheme updates the content of parity (redundant) memory banks by encoding these regions. 
The second solution involves predicting the patterns of memory addresses over time. 
%Based on this prediction, the data from free bank is prefetched to serve subsequent request for information with the help of the prefetched data. This creates the opportunities to serve a large number of access requests in a given memory clock cycle. We note that the design of such prefacing schemes crucially depends on the underlying coding scheme.
%with Accompanied the design using dynamic coding where data is moved between coded and uncoded states. Utilized the coded memory system to perform useful data prefetching.
\item Finally, we conduct a detailed evaluation of the proposed designs of shared memory systems in Section~\ref{sec:experimentalmethodology}. We implement our memory designs by extending Ramulator, a DRAM simulator.~\cite{Ramulator}. We use the gem5 simulator~\cite{parsec_2_1_m5} to create memory traces of the PARSEC benchmarks~\cite{bienia09parsec2} which are input to our extended version of Ramulator. We then observe the execution-time speedups our memory designs yield.%a Implementation of the proposed solution using system C. Performance evaluation of the proposed solution on real memory traces with the help the system C implementation.  evaluate each
%of them for their cost. We also implement these designs using systemC and regress it throughmemory traces from real multi-core system.
\end{itemize}

%problem of concentrated accesses to a particular bank by normalizing it across 
%several banks. The solution is to use coding theory techniques to create 
%redundancy across banks, increasing the number of parallel accesses per cycle.  
%The queue build up on a bank is serviced through parallel access to several 
%additional banks, known as parity banks. The additional bank accesses results in 
%a decrease in number of contended memory accesses between cores, therefore 
%reducing the overall latency of the system. The reduction in the latency can be 
%seen directly as an increase in the overall system performance. 
%We present various design to store the redundancy across the parity banks and evaluate each
%of them for their cost. We also implement these designs using systemC and regress it through
%memory traces from real multi-core system.
%We show that 
%with a memory overhead of 15 $\%$; we can enable 4 extra read accesses / 2 extra 
%write accesses to a bank while remaining within the given design parameters. 

%{\color{blue}
%\subsection{Main contributions}
%
%Here, we summarize the main contributions of this paper. 
%\begin{itemize}
%\item Taking a coding theoretic approach to address the issue of realizing multi-port memories from single port memories. 
%\item Designed the memory controller accordingly:
%\begin{itemize}
%\item Involves devising scheduling schemes for both read and write requests.
%\end{itemize}
%\item Accompanied the design using dynamic coding where data is moved between coded and uncoded states. 
%\item Utilized the coded memory system to perform useful data prefetching.
%\item Implementation of the proposed solution using system C. Performance evaluation of the proposed solution on real memory traces with the help the system C implementation. 
%\end{itemize}
%}

%\noindent \textbf{Organization:~} The rest of the paper is organized as follows.

%\textbf{Key issues that need to be addressed}

%\begin{itemize}
%\item Design of storage space, i.e., how the data is distributed among different memory banks. This includes generation of redundancy (parity bits) based on the original data and allocation of these parity bits to the memory banks.
%\item Keeping track of the validity of the data stored in various banks. Note that, due to unserved previous write requests, some of the stored data might have become outdate as far as a particular read request is concerned.
%\item Resource allocation/arbitration among the different read and write requests originated from the same or different processors. The arbitration mechanism should take multiple criterion into account, including performance (i.e., the latency viewed by the processors), efficient utilization of the storage space (i.e., minimize the unused memory bank during a given period of time), and fairness (i.e., no processor should unnecessarily suffer due to requests from other processors getting prioritized).
%\end{itemize}

%%%%%%%%%%%%

%%%%%%%%%%%%%%%%%%%%%%%%%%%%%%%%%%%%%%%%%%%%%%%%%%%%%%%%
% Background and Related Work
%%%%%%%%%%%%%%%%%%%%%%%%%%%%%%%%%%%%%%%%%%%%
\section{Background and Related Work}
%\section{Preliminaries and Related Work}
\label{sec:bg}

\subsection{Emulating multi-port memories}
\label{sec:emulation}

Multi-port memory systems are essential for multi-core computation. Individual cores may request memory simulateously, and absent a multi-port memory system some cores will stall. Designing multi-port memory system has significant costs. Complex circuitry and large area requirements for multi-port bit-cells are significantly higher than those for single-port bit-cells~\cite{Suzuki,WLCH14}. This motivates the exploration of algorithmic and systematic designs that emulate multi-port memories using single-ported memory banks~\cite{ACP88, EMY91, RG91,Memoir_xor, Memoir_xor_virtual}. Attempts have been made to emulate multi-port memory using replication based designs \cite{CCES93}, however the resulting memory architectures are very large. \Ethan{Move some of this earlier?}

%\Ankit{This patent by  Chappell, Chappell, Ebcioglu and Schuster\cite{CCES93} arguing against both multi-port RAMs and their emulation using single-port RAMs. But the emulation is replication based so we can make a case here for coding based emulation.....}

\subsubsection{Read-only Support} 
\label{sec:read_only}
Replication-based designs are often proposed as a method for multi-port emulation. Suppose that a memory design is required to support only read requests, say $r$ read requests per memory clock cycle. A simple solution is storing $r$ copies of each data element on $r$ different single-port memory banks. In every memory clock cycle, the $r$ read requests can be served in a straightforward manner by mapping all read request to distinct memory banks (see Figure~\ref{fig:read_replication}). This way, the $r$-replication design completely avoids bank conflicts for up to $r$ read request in a memory clock cycle. 

\begin{remark}
\label{rem:read_only}
If we compare the memory design in Figure~\ref{fig:read_replication} with that of Figure~\ref{fig:example_xor}, we notice that both designs can simultaneously serve $2$ read requests without causing any bank conflicts. Note that the design in Figure~\ref{fig:example_xor} consumes smaller storage space as it needs only $3$ single-port memory banks while the design in  Figure~\ref{fig:read_replication} requires $4$ single-port memory banks. However, the access process for the design in Figure~\ref{fig:example_xor} involves some computation. {\color{red}This observation raises the notion that sophisticated coding schemes allow for storage efficient designs compared to replication based methods~\cite{MacSlo}. However, this comes at the expense of increased computation required for decoding.}
\end{remark}

%---------------------------
\begin{figure}[t!]
\centering
\includegraphics[width=0.425\linewidth]{fig/read-replication.pdf}
\caption{A $2$-replication design which supports $2$ read requests per bank. In this design, the data is partitioned between two banks $\mathbf{a} = [a(1),\ldots, a(L)]$ $\mathbf{b} = [b(1),\ldots, b(L)]$ and duplicated.}
\label{fig:read_replication}
\end{figure}
%---------------------------

%---------------------------
\begin{figure}[t!]
\centering
\includegraphics[width=0.86\linewidth]{fig/rw-replication.pdf}
\caption{A $4$-replication based design to support $r = 2$ read requests and $w = 1$ write requests. Both collections of information elements $\mathbf{a} = [a(1),\ldots, a(L)]$ and $\mathbf{b} = [b(1),\ldots, b(L)]$ are replicated to obtain $r\cdot (w + 1) = 4$ single-port memory banks. These banks are then partitioned into $r = 2$ disjoint groups, Banks $1$ -- $4$ and Banks $5$ -- $8$. 
Suppose that there are two read requests for $\{a(i), a(j)\}$ and a write request for $\{a(k)\}$. The memory architecture here enables the memory controller to schedule all three requests targeting bank $\mathbf{a}$ in the same memory cycle. The two read requests are served using one instance of bank $\mathbf{a}$ in each of the disjoint groups. The write request is served to one of the instances of $\mathbf{a}$ in each group. We serve the write requests to each disjoint group to ensure that each group contains up-to-date data. Each group must contain up-to-date data so that any arbitrary set of 2 read requests can be served by the two groups. As the write is served, the pointer storage is updated to keep track of the state of the data in the banks.}
\label{fig:rw_replication}
\end{figure}
%---------------------------
\subsubsection{Read and Write Support}
\label{sec:rw}
\Ankit{Mainly describing the results from the work of Auerbach, Chen, and  Paul\cite{ACP88}.} 
% It is evident from the discussion so far that we can indeed emulate the behavior of a multi-port memory on read requests by storing data on single-port memory banks in a redundant manner. 
A proper emulation of multi-port memory must be able to serve  write requests. A challenge that arises from this requirement is tracking the state of memory. In replication-based designs where original data banks are duplicated, the service of writes requests results in differences in state between the original and duplicate banks.

Replication-based solutions to the problems presented when supporting write requests involve creating yet more duplicate banks. A replication-based multi-port memory emulation that simultaneously supports $r$ read requests and $w$ write requests requires a $r\cdot(w + 1)$ replication scheme, where $r\cdot(w+1)$ copies of each data element are stored on $r\cdot(w + 1)$ different single-port memory banks. We illustrate this scheme for $r = 2$ and $w = 1$ in Figure~\ref{fig:rw_replication}. As in previous illustrations, we have two groups of symbols $\mathbf{a} = [a(1),\ldots, a(L)]$ and $\mathbf{b}  = [b(1),\ldots, b(L)]$. We store $4$ copies each of data elements $\mathbf{a}$ and $\mathbf{b}$ and partition the banks into $r = 2$ disjoint groups. Each group contains $(w + 1) = 2$ memory banks. An additional storage space, the pointer storage, is required keep track the state of the data in the banks.


%\subsection{Better emulation of multi-port memories}
\subsection{Storage-efficient emulation of multi-port memories}
\label{sec:efficient_emulation}

As described in Section~\ref{sec:emulation}, introducing redundancy to systems which use single-port memory banks allows such systems emulate the behavior of multi-port banks. In a setup where multi-port reads are supported (cf. Section~\ref{sec:read_only}) such emulation has little computational and storage cost. Emulating multi-port read and write systems is more costly (cf. Section~\ref{sec:rw}). A greater number of single-port memory banks are needed, and systems which redundantly store memory require tracking of the various versions of the data elements present in the memory banks. Furthermore, the presence of varying version of elements in the banks complicates the process of arbitration, as some memory banks may contain stale elements. Many programs in multi-core environments involve significant numbers of write requests, so any system which emulates multi-port memory using single-port memory must take these complications into account.

{\color{red}We believe that various tasks that arise in the presence of write requests and contribute to computational overhead of the memory design, including synchronization among memory banks and complicated arbitration, can be better managed at the algorithmic level.\Ethan{good point!} Note that these tasks are performed by the memory controller. It is possible to mitigate the effect of these tasks on the overall performance of memory system by relying on the increasing available computational resources while designing the memory controller. Additionally, we believe that large storage overhead is a more fundamental issue that needs to be addressed before the emulation of the multi-port memories is feasible. In particular, the large replication factor in a naive emulation create so large a storage overhead that the resulting area requirements of such designs are impractical.}

In order to reduce the storage overhead incurred by multi-port emulation, we avoid the native $r\cdot(w + 1)$-replication design. Another approach arises from the observation that some data banks are left unused during arbitration in individual memory cycles, while other data banks receive multiple requests. We encode the elements of the data banks using specific coding schemes to generate parity banks. Elements drawn from multiple data banks are encoded and stored in the parity banks. This approach allows us to utilize idle data banks to decode elements stored in the parity banks in service of multiple requests which target the same data bank. We recognize that this approach leads to increased complexity at the memory controller. {\color{red} {\em However, we show that the increase in complexity can be kept within an acceptable level while ensuring storage-efficient emulation of multi-port memories.}}

\subsection{Related work}

Coding theory is a well-studied field which aims to mitigate the challenges of underlying mediums in information processing systems ~\cite{MacSlo, Cover}. The field has enabled both reliable communication across noisy channel and reliability in fault-prone storage units. Recently, we have witnessed intensive efforts towards the application of coding theoretic ideas to design large scale distributed storage systems\cite{Azure, SAPDVCB13, Rashmi14}. In this domain of coding for distributed storage systems, the issue of access efficiency has also received attention, especially the ability to support multiple simultaneous read accesses with small storage overhead~\cite{batchcodes, RPDV16, RSDG16, Wang2017}. In this paper, we rely on such coding techniques to emulate multi-port memories using single-port memory banks. We note that the existing work on batch codes~\cite{batchcodes} focuses only on the read requests, but the emulation of multi-port memory must also handle write requests. 

Coding schemes with low update complexity that can be implemented at the speed memory systems require have also been studied ~\cite{ASV10, MCW14}). Our work is distinguished from the majority of the literature on coding for distributed storage, because we consider the interplay between read and write requests and how this interplay effects memory access latency.

In this paper, we also explore the idea of proactively prefetching the information from memory banks to improve the access efficiency of our memory design. The idea of prefetching in realizing fast data transfer between processors and memory has been previously explored in the literature (see \cite{Kim2016, Kadjo2014, Shevgoor2015, JL2013} and references therein). 
%However, our work addresses the issue of data prefetching in the context of coded memory system which is not addressed earlier in the literature. 
More recently, an LSTM-based recurrent neural network was used to predict future memory access requests on the SPEC 2006 benchmark dataset \cite{lstm2018}. This deep learning method may be used in addition to our proposed frequency-based approach.
Our combination of coded memory and prefetching also shares some similarity with the recent line of work on coded caching~\cite{MN16a} which aims to reduce the data downloaded from servers in a communication network by utilizing the cache available at the end users. Here, we would like to point out that there are many key differences in the our setup with coded memory banks with that considered in \cite{MN16a}. Our setup has data stored in an encoded form stored across memory banks and caching is enabled by the memory controller, which is a centralized unit. In contrast, the setup of coded caching has a centralized storage system (server) and cache units that store encoded information distributed across users.

{\color{red} {\bf The work which is closest to our solution for emulating a multi-port memory is by Iyer and Chuang~\cite{Memoir_xor, Memoir_xor_virtual}, where they also employ XORing based coding schemes to redundantly store information in an array of single-port memory banks. However, we note that our work significantly differers from \cite{Memoir_xor, Memoir_xor_virtual} as we specifically rely on different coding schemes arising under the framework of batch codes~\cite{batchcodes}. Additionally, due to the employment of distinct coding techniques, the design of memory controller in our work also differs from that in \cite{Memoir_xor, Memoir_xor_virtual}.}}

\Ankit{Also cite the work by Rivest et al.~\cite{RG91} and Endo, Matsumura and Yamada~\cite{EMY91}.}

%\section{Motivation}
%
%\subsection{Dual port RAM}
%
%\subsubsection{Replication}
%Because the size of the dual-ported SRAM bit-cell is almost double that of the single-ported SRAM bit-cell, a more versatile way (i.e., can do 2 reads in one cycle or one write) to implement dual-ported SRAM is by duplicating SRAM banks, as shown in Figure 3. In this way the bandwidth does not suffer from any loss for performing 2 simultaneous read operations, but only suffers 1 arbitration loss when performing simultaneous read and write operations (i.e., cannot perform 1 read and 1 write in the same cycle). The area is similar to that used in dual-ported circuit implementations. The advantage of this implementation is simplicity while the disadvantage is that if frequent write access is required the performance (bandwidth) is not as good as the true 1R1W SRAMs which can do a simultaneous read and write in one cycle.
%
%\subsubsection{Replication with pointer storage}
%
%Replication scheme $r + w$ replications...for $rRwW$ multipart memory.
%
%\subsubsection{Bank interleaving and arbitration}
%
%Another alternative is to use bank interleaving and arbitration circuits to allow for simultaneous access of different banks of memory. Occasional stalls are necessary in this approach if the arbitration circuit finds conflicting access to the same memory banks. Its operation is illustrated in Figure 4.
%
%Figure 4. Multiple accesses to SRAM by Bank interleaving with lower address bits. A large SRAM bank is sub-divided into n (n ? 2) smaller single-ported SRAM banks to support 2 or more simultaneous accesses. An arbitration unit is used in case conflicting addresses want to access the same bank, in which case one of the accesses is delayed.
%
%
%
%
%{\color{red}
%SRAMs can be categorized as single-ported or multi-ported. The single-ported SRAM is the most common type of SRAM with the best area efficiency and is used for most compiler memories due to its modular approach. Multi-ported memories are either not area efficient or with limited storage capability. Logical implementation of multi-ported memories with interleaved banks can be area efficient (because there is no memory duplication), but requires arbitration and flow-control circuits. Duplicating single-ported memory banks can support 2 reads and 1 write type of accesses with no additional delays and 1 read and 1 write type of access with 1.5 times the delay of read access;  Also its size is competitive relative to the true dual-ported memories. True multi-ported memories can be implemented with single-ported memories by replicating the memories multiple times. The area cost is r(w+1) times the number of bank replications.  Algorithmic approaches of multi-port memory design (such as Memoir) include caches and different ways of buffering of read and write data, with advantages being area efficiency and disadvantages including design complexity. Algorithmic memories can be statistical with lower areas or deterministic with higher areas, depending on the application.}



%%%%%%%%%%%%%%%%%%%%%%%%%%%%%%%%%%%%%%%%%%%%%%%%%%%%%%%%
% Code design
%%%%%%%%%%%%%%%%%%%%%%%%%%%%%%%%%%%%%%%%%%%%
\section{Codes to Improve Accesses}
\label{sec:code_design}

%In this section, we discuss one of the main components of the memory design proposed in this paper. 
Introducing redundancy into a storage space comprised of single-port memory banks enables simultaneous memory access. In this section we propose memory designs that utilize coding schemes which are designed for access-efficiency. We first define some basic concepts with an illustrative example and then describe $3$ coding schemes in detail.

\begin{comment}
Coding theory is the study of codes and their applications  to specific fields.  
Coding has been used in a variety of computer  science applications,  from error 
correction in the transmission of data  to increased compression for data  
storage.  We aim to extend the benefits of coding theory  to improve the 
efficiency of random-access  memory systems.  We propose a memory scheme in 
which a small portion of memory is reserved for the efficient coding of 
pre-existing data.  In essence, this allows the data of one bank to be 
duplicated and stored in an additional memory location.  Traditionally, when 
multiple  requests  to a single bank  are issued by the processor, a stall is 
generated.  These types of stalls, known as bank conflicts, result from the fact 
that  only one address from a single bank can be accessed at a time.  The 
processor must wait for the result from the first bank access to return  before 
it can serve additional  requests to the same bank.  This lag can be a major 
bottleneck in a computer's processing speed. With a coded memory scheme, data 
present in multiple data banks will be compressed and stored in extra banks, 
known as a parity banks.  These parity banks will then be accessed concurrently 
with corresponding data  banks to help alleviate stalls from bank conflicts.  
Ultimately,  with the addition  of a single parity  bank we are able to generate 
a single additional  access to any arbitrary bank  without  implementing  any 
further  logic to the bank  itself.  In the following sections, we first 
describe the design parameters used to design the coding system.  We then 
describe each of the three code designs explored in this project.
\end{comment}

\subsection{Coding for memory banks}
\label{sec:coding_mb}

A coding scheme defines how memory is encoded to yield redundant storage. The memory structures which store the original memory elements are known as {\em data banks}. The elements of the data banks go through an {\em encoding process} which generate a number of {\em parity banks}.  The parity banks contain elements constructed from elements drawn from two or more data banks. A linear encoding process such as XOR may be used to minimize computational complexity. The following example further clarifies these concepts and provides some necessary notation.

\begin{example}
Consider a setup with two data banks $\mathbf{a}$ and $\mathbf{b}$. We assume that each of banks store $L \cdot W$ binary data elements\footnote{It is possible to work with data elements over larger alphabets/finite fields. However, assuming data elements to be binary suffices for this paper as only work with coding schemes defined over binary field.} which are arranged in an $L \times W$ array. In particular, for $i \in [L] \triangleq \{1,\ldots, L\}$, $a(i)$ and $b(i)$ denote the $i$-th row of the bank $\mathbf{a}$ and bank $\mathbf{b}$, respectively. Moreover, for $i \in [L]$ and $j \in [W] \triangleq \{1,\ldots, W\}$, we use $a_{i, j}$ and $b_{i, j}$ to denote the $j$-th element in the rows $a(i)$ and $b(i)$, respectively. Therefore, for $i \in [L]$, we have 
\begin{align}
a(i) = \big(a_{i,1}, a_{i,2},\ldots, a_{i, W}\big) \in \{0, 1\}^W\nonumber \\
b(i) = \big(b_{i,1}, b_{i,2},\ldots, b_{i, W}\big) \in \{0, 1\}^W. \nonumber
\end{align}
Now, consider a linear coding scheme that produces a parity bank $\mathbf{p}$ with $L'W$ bits arranged in an $L' \times W$ array such that for $i \in [L'] \triangleq \{1,\ldots, L'\}$, 
\begin{align}
p(i) &= \big(p_{i, 1},\ldots, p_{i,W}\big) = a(i) + b(i) \nonumber \\
&\triangleq \left(a_{i,1} + b_{i,1}, a_{i,1} + b_{i,1},\ldots, a_{i,1} + b_{i,1}\right). 
\end{align}
\end{example}
\begin{remark}
Figure~\ref{fig:example1} illustrates this coding scheme. Because the parity bank is based on those rows of the data banks that are indexed by the set $[L'] \subseteq [L]$, we use the following concise notation to represent the encoding of the parity bank. 
$$
\mathbf{p} = \mathbf{a}([L']) +  \mathbf{b}([L']).
$$
In general, we can use any subset $\mathcal{S} = \{i_1, i_2,\ldots, i_{L'}\} \subseteq [L]$ comprising $L'$ rows of data banks to generate the parity bank $\mathbf{p}$. In this case, we have $\mathbf{p} = \mathbf{a}(\mathcal{S}) +  \mathbf{b}(\mathcal{S})$, or
\begin{align*}
p(l) = a(i_l) + b(i_l)~\text{for}~l \in [L'].
\end{align*}
%Figure~\ref{fig:example1_case2} illustrates the case with a generic set $\mathcal{S}$.% $ = [L - L'  + 1,\ldots, L]$.
\end{remark}

%%%%%%%%%%%%%%%%%%%%%%%%%%%%%%%%%%%
%\begin{figure}[t!]
%\centering
%\begin{subfigure}[b]{0.48\linewidth}
%  \centering
%  \includegraphics[width=0.95\linewidth]{fig/example-inter-bank.pdf} 
%  \caption{{\color{red}Parity.}}
%  \label{fig:example1_case1}
%\end{subfigure}
%\begin{subfigure}[b]{0.48\linewidth}
%  \centering
%  \includegraphics[width=0.98\linewidth]{fig/example-inter-bank-2.pdf} 
%  \caption{{\color{red}Parity.}}
%  \label{fig:example1_case2}
%\end{subfigure}
%\caption{{\color{red}Design.}}
%\label{fig:example1}
%\end{figure}
\begin{figure}[t!]
\centering
  \includegraphics[width=0.45\linewidth]{fig/example-inter-bank.pdf} 
\caption{\it{Notation of example parity design.}}
\label{fig:example1}
\end{figure}
%%%%%%%%%%%%%%%%%%%%%%%%%%%%%%%%%%%%%%


\begin{remark}
Note that we allow for the data banks and parity banks to have different sizes, \textit{i.e.} $L \neq L'$. This freedom in memory design can be utilized to reduce the storage overhead of parity banks based on the underlying application. The case when the size of a parity bank is smaller than a data bank, \textit{i.e.} $L' < L$, we say that the parity bank is a {\em shallow bank}. We note that it is reasonable to assume the existence of shallow banks, especially in proprietary designs of integrated memories in a system on a chip (SoC).
\end{remark}

\begin{remark}
\label{rem:design1}
Note that the size of shallow banks is a design choice which is controlled by the parameter $0 < \alpha \leq 1$. A small value of $\alpha$ corresponds to small storage overhead. The choice of a small $\alpha$ comes at the cost of limiting parity memory accesses to certain memory ranges. In Section~\ref{sec:dynamicCoding} we discuss techniques for choosing which regions of memory to encode. In scenarios where many memory accesses are localized to small regions of memory, shallow banks can support many parallel memory accesses for little storage overhead. For applications where memory access patterns are less concentrated, the robustness of the parity banks allows one to employ a design with $\alpha = 1$.
\end{remark}

\subsubsection{Degraded reads and their locality}
\label{sec:degraded}

The redundant data generated by a coding scheme mitigates bank conflicts by supporting multiple {\bf read} accesses to the original data elements. Consider the coding scheme illustrated in Figure~\ref{fig:example1} with a parity bank $\mathbf{p} = \mathbf{a}([L']) + \mathbf{b}([L'])$. In an uncoded memory system simultaneous read requests for bank $\mathbf{a}$, such as $a(1)$ and $a(5)$, result in a bank conflict. The introduction of $\mathbf{p}$ allows both read requests to be served. First, $a(1)$ is served directly from bank  $\mathbf{a}$. Next, $b(5)$ and $p(5)$ are downloaded. $a(5) = b(5) + p(5)$, so $a(5)$ is recovered by means of the memory in the parity bank. A read request which is served with the help of parity banks is termed as {\em degraded read}. Each degraded read has a parameter {\em locality} associated with it which corresponds to the total number of banks used to serve it. In this case, the degraded read for $a(5)$ using $\mathbf{b}$ and $\mathbf{p}$ has locality $2$.

%In order to further illustrate the notion of locality, let's consider a setup where we generate a parity bank $\mathbf{p}$ by combining three data banks $\mathbf{a}$, $\mathbf{b}$, and $\mathbf{c}$ as $\mathbf{p} = \mathbf{a} + \mathbf{b} + \mathbf{c}$. Now, a degraded read for $a(1)$ using the parity bank as $$a(1) = b(1) + c(1) + p(1) = b(1) + c(1) + \big(a(1) + b(1) + c(1)\big)$$
%has locality $3$ as the degraded read is served using three memory banks.

\subsection{Codes to emulate multi-port memory}
\label{sec:designs}

We will now describe the code designs proposed for the emulation of multi-port memories. Among a large set of possible coding schemes, we focus on three specific coding schemes for this task. We believe that these three coding schemes strike a good balance among various quantitative parameters, including storage overhead, number of simultaneous read requests supported by the array of banks, and the locality associated with various degraded reads. Furthermore, these coding schemes respect the practical constraint of encoding across a small number of data banks. In particular, we focus on the setup with $8$ memory banks (the design scales with larger number of banks). This contrasts with the communications applications where encoding typically occurs with blocks of $1024$ or more information symbols. 

In the rest of this section, we present three code designs and discuss the number of simultaneous read requests supported by these designs in the best and worst case. We also summarize all the relevant parameters associated with these designs in Table~\ref{table:codedesigncomparison}.
%
%We discuss the design of the codes for creating extra accesses to memory in this 
%section. First we discuss the code designs explored during Phase I. Second, we discuss 
%specific execution strategies to efficiently implement the designs.\\
%In the following sub-sections, we discuss 3 designs for storing coded data.  
%Table~\ref{table:codedesigncomparison} compares these designs for various 
%parameters and associated costs.  
%\begin{table*}[t]
%\centering
%	\begin{tabular}{|m{1cm}|m{2 cm}|m{1cm}|m{1cm}|m{1cm}|m{1cm}|m{1cm}|}
%\hline
%Design & Max Read per bank & Max Write per bank & Locality & Rate & Memory 
%Overhead & Logical Complexity \\ \hline
%I & 4 & 2 & 2 & $2/5$ & 1.5 $\alpha$ & Low \\ \hline
%II & 5 & 2 & 2 & $2/5$ & 2.5 $\alpha$ & Medium \\ \hline
%III & 4 & 2 & 3 & $1/2$ & \text{      } $\alpha$ & Medium \\ \hline
%	\end{tabular}
%	\caption{Comparison of design with respect to the performance parameters 
%	and associated cost}
%	\label{table:codedesigncomparison}
%\end{table*}


%\begin{tiny}
%\begin{table}[t!]
%  \centering
%  \begin{tabular}{|c|c|c|c|c|c|c|}
%    \hline
%    \textbf{Design} & \textbf{Max reads} & \textbf{Max writes} & \textbf{Locality} & \textbf{Rate} & \textbf{Storage overhead} & \textbf{Logical complexity} \\
%    & \textbf{(per bank)} & \textbf{(per bank)} & & & & \\
%    \hline
%    \hline
%    I & 4 & 2 & 2 & $2/5$ & 1.5 $\alpha$ & Low \\ \hline
%II & 5 & 2 & 2 & $2/5$ & 2.5 $\alpha$ & Medium \\ \hline
%III & 4 & 2 & 3 & $1/2$ & \text{      } $\alpha$ & Medium \\ 
%\hline                                   
%  \end{tabular}
%	\caption{Comparison of the code designs with respect to the performance parameters and associated cost}
%	\label{table:codedesigncomparison}
%\end{table}
%\end{tiny}

\begin{comment}
\begin{table}[t!]
  \centering
  \begin{tabular}{|c|c|c|c|c|c|}
    \hline
   {\small Design} & {\small  Max reads} &{\small  Locality} & {\small  Rate} & {\small  Storage} & {\small  Logical } \\
    & {\small  (per bank)} & & &{\small  overhead} & {\small  complexity} \\
    \hline
    \hline
    {\small I} & {\small$4$ } & {\small$2$} & {\small ${2}/{5}$} & {\small $1.5\alpha$} & {\small Low} \\ \hline
{\small II} & {\small$5$}  & {\small$2$} & {\small ${2}/{5}$} & {\small $2.5\alpha$} & {\small Medium} \\ \hline
{\small III }& {\small$4$}  & {\small$3$} & {\small$1/2$} & {\small $\alpha$} & {\small Medium} \\ 
\hline                                   
  \end{tabular}
	\caption{\it{Comparison of the code designs with respect to the performance parameters and associated cost. $\alpha$ is the fraction of storage overhead in comparison to the data bank. $\alpha = 1$ when size of parity bank is equal to size of data bank.}}
	\label{table:codedesigncomparison}
\end{table}
\end{comment}

\begin{table}[ht!]
  \centering
  \begin{tabular}{|c|c|c|c|c|c|}
    \hline
   {\scriptsize Design} & {\scriptsize  Max reads} &{\scriptsize  Locality} & {\scriptsize  Rate} & {\scriptsize  Storage} & {\scriptsize  Logical } \\
    & {\scriptsize  (per bank)} & & {\scriptsize ($\alpha=1$)}&{\scriptsize  overhead} & {\scriptsize  complexity} \\
    \hline
    \hline
    {\scriptsize I} & {\scriptsize$4$} & {\scriptsize$2$} & {\scriptsize $\nicefrac{2}{5}$} & {\scriptsize $1.5\alpha$} & {\scriptsize Low} \\ \hline
%{\scriptsize II} & {\scriptsize$5$}  & {\scriptsize$2$} & {\scriptsize ${2}/{5}$} & {\scriptsize $2.5\alpha$} & {\scriptsize Medium} \\ \hline
{\scriptsize II} & {\scriptsize$5$}  & {\scriptsize$2$} & {\scriptsize $\nicefrac{2}{7}$ & {\scriptsize $2.5\alpha$} & {\scriptsize Medium} \\ \hline
{\scriptsize III }& {\scriptsize$4$}  & {\scriptsize$3$} & {\scriptsize$\nicefrac{1}{2}$} & {\scriptsize $\alpha$} & {\scriptsize Medium} \\ 
\hline                                   
  \end{tabular}
	\caption{Comparison of the code designs with respect to the performance parameters and associated cost}
	\label{table:codedesigncomparison}
\end{table}


\subsubsection{Code Design I}
\label{sec:design1}

This code design is motivated from the concept of batch codes~\cite{batchcodes} which enable parallel access to the content stored in a large scale distributed storage systems.
%The coding scheme is illustrated in Figure~\ref{fig:design1}. 
The code design involves $8$ data banks $\{\mathbf{a}, \mathbf{b},\ldots, \mathbf{h}\}$ each of size $L$ and $12$ shallow banks each of size $L' = \alpha L$. We partition the $8$ data banks into two  groups of $4$ banks. The underlying coding scheme produces shallow parity banks by separately encoding data banks from the two groups. Figure~\ref{fig:design1} shows the resulting memory banks. The storage overhead of this design is $12\alpha L$ which amounts to the rate\footnote{The information rate is a standard measure of redundancy of a coding scheme ranging from $0$ to $1$, where $1$ corresponds to the most efficient utilization of storage space.} of the coding scheme is $$\frac{8L}{8L + 12\alpha L} = \frac{2}{2 + 3\alpha}.$$


We now analyze the number of simultaneous read requests that can be supported by this code design. \\
%This allows us to serve multiple accesses to the coded 
%region using the parity banks. With this scheme, we guarantee that any 4 read 
%requests to the coded region can be served at any given time. As shown in 
%figure~\ref{fig:design1}, 8 banks are divided into two regions.  Each region 
%consists of 4 banks. Each region has 6 parallel shallow  banks to store the 
%parity. The colored regions shown in the banks 1-8 are the coded region. These 
%regions are assumed to be of $\alpha $ fraction of the memory. \\

\noindent \textbf{Best case analysis:~}This code design achieves maximum 
performance when sequential accesses to the coded regions are issued. During the 
best case access, we can achieve up to $10$ parallel accesses to a particular coded region in one access cycle.
Consider the scenario when we receive accesses to the following $10$ rows:
\begin{align*}
&\left\{a(1),b(1),c(1),d(1),a(2),b(2),c(2),d(2),c(3),d(3)\right\} .
\end{align*}
Note that we can serve the read requests for the rows \\ $\{a(1),b(1),c(1),d(1)\}$ using the data bank $\mathbf{a}$ and the three parity banks storing $\{a(1)+b(1), b(1)+c(1),c(1)+d(1)\}$. The requests for $\{a(2),c(2),d(2)\}$ can be served by downloading $b(2)$ from the data bank $\mathbf{b}$ and $\{a(2)+d(2), b(2)+d(2),a(2)+c(2)\}$ from the respective parity banks storing these. Lastly, in the same memory clock cycle, we can serve the requests for $\{c(3), d(3)\}$ using the data banks $\mathbf{c}$ and $\mathbf{d}$.\\
%------------------------------
\begin{figure}[ht!]
\centering
	%\includegraphics[width=0.8\linewidth]{fig/designI.pdf}
	\includegraphics[width=1\linewidth]{fig/Code-Design-1.pdf}
	\caption{\it{Code Design I}}
	\label{fig:design1}
%\caption{Code Designs}
\end{figure} 
%------------------------------
\ignore{
%------------------------------
\begin{figure}[ht!]
\centering
\includegraphics[width=150mm,natwidth=610,natheight=642]{fig/result_design1.jpg}
\caption{ }
\label{fig:result_design1}
\end{figure}
%------------------------------
}
\noindent \textbf{Worst case analysis}: This code design  (cf.~Figure~\ref{fig:design1}) may fail to utilize any parity banks depending on the requests waiting to be served. The worst case scenario for this code design occurs when there are non-sequential and non-consecutive access to the memory 
banks. Take for example a scenario where we only consider the first four banks of the code design. The following read requests are waiting to be served:  
\begin{align*}
\{a(1), a(2), b(8), b(9), c(10),c(11), d(14), d(15)\}. 
\end{align*}
Because none of the requests share the same row index, we are unable to utilize the parity banks. However, we still benefit from the prefetching mechanism discussed in Section~\ref{sec:prefetching}. The worst case number of reads per cycle is equal to the number of data banks. 
%In Figure~\ref{fig:result_design1} , we explore the worst case scenario when 
%the accesses are random. The results show that the queue build up for reads and 
%writes does fall back to no-coding scenario. This asserts that the worst case 
%scenario for a coding scheme performs similar to no-coding scheme.In the second 
%scheme, we augment the code storage by cross storing the codes from region 1 to 
%region 2 and vice-versa.We do this in addition to coding the consecutive memory 
%addresses in a bank. This provides two benefits, first it increases the overall 
%redundancy, and second it allows us to use the parity banks of the other region 
%in case the first region�s parity banks are in use. 
\ignore{
\begin{figure}[!ht]
\centering
\includegraphics[width=150mm,natwidth=610,natheight=642]{fig/result_design2.jpg}
\caption{ Comparison of Design II with No coding case }
\label{fig:result_design2}
\end{figure}
}
\subsubsection{Code Design II}
\label{sec:design2}

Figure~\ref{fig:design2} illustrates the second code design explored in this paper. Again, the $8$ data banks $\{\mathbf{a}, \mathbf{b},\ldots, \mathbf{h}\}$ are partitioned into two groups containing $4$ data banks each. These two groups are then associated with two code regions. The first code region is similar to the previous code design, as it contains parity elements constructed from two data banks. The second code region contains data directly duplicated from single data banks. This code design further differs from the previous code design (cf. Figure~\ref{fig:design1}) in terms of the size and arrangement parity banks. Even though $L' = \alpha L$ rows from each data bank are stored in a coded manner by generating parity elements, the parity banks are assumed to be storing $2\alpha L > L'$ rows.

For a specific choice of $\alpha$, the storage overhead of this design is $20\alpha L$ which amounts to the rate of the coding scheme being $$\frac{8L}{8L + 20\alpha L} = \frac{2}{2 + 5\alpha}.$$ Note that this code design can support $5$ read accesses per data bank in a single memory clock cycle as opposed to $4$ read requests supported by the code design from Section~\ref{sec:design1}. However, this is made possible at the cost of extra storage overhead. Next, we discuss the performance of this code design in terms of the number of simultaneous read requests that can be served in the best and worst case.

%
%The second design, presented in figure 4, improves over first design by allowing 
%5 read accesses per bank per cycle. This design also divides banks into two 
%regions. The first region is
%Bank 1 to Bank 4 and 5 corresponding Parity banks. The two regions in figure 4 
%are upper 9 banks forming one region and lower 9 banks forming another. This 
%design allows intermix storage of parity among regions. The design uses 5 parity 
%banks per region. The data in this scheme is coded for both inter bank and 
%intra-bank. The intra-bank codes are stored in the alternate parity bank region. 
%This allows usage of parity banks from other region if they are available. \\
\begin{figure}[!ht]
%\centering
%\begin{minipage}[!t]{\linewidth}
	%\includegraphics[width=1\linewidth]{fig/designII.pdf}
	\includegraphics[width=1\linewidth]{fig/Code-Design-2.pdf}
	\caption{\it{Code Design II}}
	\label{fig:design2}
%\end{minipage}
\end{figure}

\noindent \textbf{Best case analysis:~} This code design achieves the best access performance when sequential accesses to the data banks are issued. In particular, this design can support up to $9$ read requests in a single memory clock cycle. Consider the scenario where we receive read requests for the following rows of the data banks. 
$$
\big\{a(1),b(1),c(1),d(1),a(2),b(2),c(2),d(2),a(3),b(3),c(3)\big\}
$$ Here, we can serve 
$\{a(1), b(1), c(1), d(1)\}$ using the data bank $\mathbf{a}$ with the parity banks storing the parity elements $\{a(1) + b(1),b(1)+c(1),c(1)+d(1)\}$. Similarly, we can serve the requests for the rows $\{a(2),b(2),d(2)\}$ using the data bank $\mathbf{b}$ with the parity banks storing the parity elements $\{a(2)+d(2), b(2)+d(2)\}$. Lastly, the request for the rows $c(2)$ and $d(3)$ is served using the data banks $\mathbf{c}$ and $\mathbf{d}$.\\


\noindent \textbf{Worst case analysis:~}The code scheme can enable $5$ simultaneous accesses in a single memory clock cycle in the
worst case. These are non-sequential and non-consecutive accesses to the memory banks. For 
example, when the access pattern corresponds to the rows $\{a(1),b(6),c(9),d(15),e(20)\}$, we can simultaneously serve 
these $5$ read requests with the help of our coded memory. In order to better utilize the unused banks in this case, we can use the prefetching 
mechanisms (cf. Section~\ref{sec:prefetching}) to look ahead in the queue and proactively download elements from the unused banks for future accesses.


% This design employs both inter-bank and intra-bank encoding in order to generate the content to be stored on the parity banks. In order to illustrate another flexibility that can be utilized while designing the storage space for a memory system, even though this code design encodes $\alpha L$ rows from each data bank, the parity banks are assumed to be storing $2\alpha L$ rows.


\subsubsection{Code Design III}
The next code design we discuss has locality 3, so each degraded read requires two parity banks to be served. This code design works with $9$ data bank $\{\mathbf{a}, \mathbf{b},\ldots, \mathbf{h}, \mathbf{z}\}$ and generates $9$ shallow parity banks. Figure~\ref{fig:design3} shows this design.
%The two designs discussed above achieve a rate of $2/5$. Here, we explore a code design which achieves a rate of $1/2$. 
%This design requires 9 data banks and 9 parity banks as shown in figure 5. It 
%also has a comparatively higher locality of 3. That is, it requires the memory 
%controller to "know" two out of three data elements to decode the third. 
The storage overhead of this design is $9\alpha L$ which corresponds to the rate of $\frac{1}{1 + \alpha}$. We note that this design possesses higher logical complexity as a result of its increased locality. 

This design supports $4$ simultaneous read access per bank per memory clock cycle as demonstrated by the following example. Suppose there are requests for rows $\{a(1), a(2), a(3), a(4)\}$. $a(1)$ can be served directly from $\mathbf{a}$. $a(2)$ is served by means of a parity read and reads to banks $\mathbf{b}$ and $\mathbf{c}$, $a(3)$ is served by means of a parity read and reads to banks $\mathbf{d}$ and $\mathbf{g}$, and $a(4)$ is served by means of a parity read and reads to banks $\mathbf{e}$ and $\mathbf{z}$.

\noindent \textbf{Best case analysis:~} Following the analysis similar to design I and II, the best case number of reads per cycle will be equal to the number of data and parity banks.

\noindent \textbf{Worst case analysis:~} Similar to design I and design II, the number of reads per cycle is equal to the number of data banks. 
%The memory overhead here is less (just $\alpha$) compared to the previous designs. However, it possesses higher logical complexity because of increased locality. Example cases for this design are described below :
%\begin{itemize}
%
%	\item 4 reads for $a_0$: 1 read from $a_0$, 1 read from ($a_1$, $a_2$, 
%		$a_0$ + $a_1$ + $a_2$), 1 read from ($a_3$, $a_6$, $a_0$ + $a_3$ 
%		+ $a_6$), and the 4th read from ($a_4$, $a_8$, $a_0$ + $a_4$ + 
%		$a_8$).
%	\item 3 reads for $a_0$: 1 read from $a_0$, 1 read from ($a_3$, $a_6$, 
%		$a_0$ + $a_3$ + $a_6$), and the 3rd read from ($a_4$, $a_8$, 
%		$a_0$ + $a_4$ + $a_8$). \\
%	      1 read for $a_1$:  1 read from $a_1$.
%	\item 2 reads for $a_0$: 1 read from $a_0$ and the 2nd read from ($a_3$, 
%		$a_6$, $a_0$ + $a_3$ + $a_6$). \\
%	      2 reads for $a_1$: 1 read from $a_1$ and the 2nd read from ($a_4$, 
%	      $a_7$, $a_1$ + $a_4$ + $a_7$).
%	\item 2 reads for $a_0$: 1 read from $a_0$ and the 2nd read from ($a_3$, 
%		$a_6$, $a_0$ + $a_3$ + $a_6$). \\
%	      1 read for $a_1$: 1 read from $a_1$. \\
%	      1 read for $a_2$: 1 read from $a_2$.
%    \end{itemize}
%---------------------------------------
\begin{figure}[!ht]
	\centering
	\begin{minipage}[!t]{\linewidth}
		\includegraphics[width=\linewidth]{fig/Code-Design-3_9banks.pdf}
		\caption{\it{Code Design III}}
		\label{fig:design3}
	\end{minipage}
\end{figure}
%---------------------------------------

\begin{remark}
Note that the coding scheme in Figure~\ref{fig:design3} describes a system with $9$ data banks. However, we have set out to construct a memory system with $8$ data banks. It is straightforward to modify this code design to work with $8$ data banks $\{\mathbf{a}, \mathbf{b},\ldots, \mathbf{h}\}$ as shown in Figure~\ref{fig:design3_8}.
%Since most systems are implemented with number of banks as $2^n$ for some n. We present an 
%example of the code with 8 data banks in figure~\ref{fig:design3_8}. For using 8 data banks, 
%we drop the bank I. We also ignore the data from Bank I for constructing parity. So, three of 
%the parity banks have the locality of 2, while the rest of the parity banks have locality of 3.
% The new scheme for 8 data banks has 9 parity banks.
 \end{remark}
%---------------------------------------
\begin{figure}[!ht]
\centering
	\begin{minipage}[!t]{\linewidth}
		\includegraphics[width=\linewidth]{fig/Code-Design-3_8banks.pdf}
		\caption{\it{Code Design III with 8 data banks}}
		\label{fig:design3_8}
	\end{minipage}
\end{figure}
%---------------------------------------

\begin{comment}
Since the locality is 3 here in this design, i.e. , each parity is made up of combination of 3 data banks, we need to make sure that all three requests are in one line to be able to use the parity bank. 
For example parity bank 0 contains A+B+C. So, the following scenarios arise:
\begin{itemize}
	\item {\em Scenario I}: 1st request of A and 1st request of B are in same row. Then, we can search for a request in the same row for bank C by doing a look ahead. 
	\item {\em Scenario II}: 1st request of A and 1st request of C are in same row. Then, we can search for a request in the same row for bank B by doing a look ahead. 
	\item {\em Scenario III}: 1st request of B and 1st request of C are in same row. Then, we can search for a request in the same row for bank A by doing a look ahead. 
\end{itemize}
So, the simple pseudo code for doing this would be :\\
\begin{verbatim}
for each data bank 
    for each auxiliary bank1 of data bank
            Look ahead in auxiliary bank2 and check if 3 request in a row.
        end
    end
\end{verbatim}
Example: - \\
For {\bf data bank} to be {\bf A} \\
{\bf auxiliary bank1} goes from [B C D G A E] \\
{\bf auxiliary bank2} goes from [C B G D E A] \\
  The element A is just there in {\bf auxiliary bank1} and {\bf auxiliary bank2} to maintain the symmetry because A + E has locality of 2.
\end{comment}


%%%%%%%%%%%%%%%%%%%%%%%%%%%%%%%%%%%%%%%%%%%%%%%%%%%%%%%%
% Memory controller design
%%%%%%%%%%%%%%%%%%%%%%%%%%%%%%%%%%%%%%%%%%%%

\section{Memory Controller Design}
\label{sec:memcontrol}
\Matt{BLUE: I would not mind removing this text. RED: I will likely remove this text}
In this section, we discuss the architecture of the memory controller for the proposed memory system. This memory controller is designed to make use of the coding schemes discussed in the previous section. The architecture of the memory controller is focused on exploiting the redundant storage in the coding schemes to serve memory requests faster than an uncoded scheme. This section presents the key architectural requirements of the memory controller and potential implementations of these requirements.
%The memory design involves two key components: 1) storage space comprising of memory banks and 2) memory controller 
%The coding schemes discussed in previous section is implemented using systemC. 
%This section describes the architectural detail of how the schemes are 
%implemented using optimized algorithms. \\
%In this section, we explore the technique of dynamic coding in order to reduce 
%the memory and access overhead
%associated with the parity banks. We first discuss the scheme of dynamic coding 
%and follow it by discussing the potential benefits of prefetching the codes.\\

\subsection{Memory Controller Stages}
A general memory controller consists of three stages of processing illustrated in Figure~\ref{fig:multicore_arch}. The first stage, the {\em core arbiter}, receives memory access request from the master cores. The core arbiter then routes the requests to the proper {\em bank queue}. The bank queues are the second stage of processing, and they are responsible for storing and tracking memory requests. A memory request seeking memory located in bank $N$ will be sent to the $N$th bank queue. The {\em access scheduler} is the final stage of processing. It is responsible for scheduling the requests in the bank queues. Each memory cycle, the access scheduler generates an access pattern based on the requests present in the bank queues. The access pattern is a description of the reads or writes the memory controller will perform on the memory banks. Next, we discuss all of these three units and their functions in a greater detail.
%first level is {\em Core Arbiter }, the unit responsible for handling requests 
%from cores.  The second level is {\em Bank Arbiter} responsible for arbitrating 
%requests to banks.  The third level is {\em Access Scheduler} which schedules 
%most efficient access for each cycle.\\
% %%-----------------------
% \begin{figure}[htbp]
% \centering
% 	\includegraphics[width=0.7\linewidth]{fig/controllerArchitecture.pdf}
% \caption{
% {Architecture of Memory Controller} }
% \label{fig:pseudo-code}
% \end{figure}
% %-------------------------
\begin{itemize}
\item \textbf{Core arbiter:~}Every clock cycle, the core arbiter receives up to one request from each core which it stores in an internal queue. The core arbiter attempts to push these request to the appropriate bank queue. If in attempting to push a request the core arbiter detects that the destination bank queue is full, the controller signals that the core is busy which stalls the core. The core arbiter is also responsible for arbitration among the access requests. It arranges the requests stored in its internal queue using a two-step priority order mechanism. It arranges the request in order of QoS priority, and it arranges requests with the same QoS priority using round-robin scheduling.

\item \textbf{Bank queues:~}Each data bank has a corresponding read queue and write queue.  The core arbiter sends memory requests to the bank queues until the queues are full. In our simulations, we use a bank queue depth of 10. 

In addition to the read and write queues, there is a single queue which holds special requests such as memory refresh requests.

\item \textbf{Access scheduler:~}The access scheduler is responsible for handling interactions with the memory banks. Every memory cycle, the access scheduler chooses to serve read requests or write requests and algorithmically determines which requests in the bank queues it will schedule. The scheduling algorithms the access scheduler uses are called pattern builders. Every memory cycle, the access scheduler invokes either the read pattern builder or the write pattern builder to schedule read or write requests respectively. A key design trade-off of the pattern builder algorithms is the relationship between the complexity of the algorithm and the number of requests the algorithm schedules.
\end{itemize}

We note that the core arbiter and bank queues should not differ much from those in a traditional setup with an uncoded storage space. The access scheduler directly interacts with the memory banks including the parity banks, so it must be designed with the proposed coding schemes in mind. The rest of this section is devoted to discussing the access scheduler in detail.


%-----------------------
\begin{figure}[tbp]
\centering
\includegraphics[width=0.7\linewidth]{fig/coded_access_scheduler.pdf}
\caption{
{Access scheduler for coded memory} }
\label{fig:coded_access_scheduler}
\end{figure}
%-------------------------
\subsection{Code Status Table}
\label{sec:codeStatusTable}
The code status table keeps track of the validity of data stored the data and parity banks. The access scheduler may serve a write request using either a data or parity bank. When a write is served to a row in a data bank, any parity bank which is constructed from the data bank will contain invalid data its corresponding row. Similarly, when the access scheduler serves a write to a parity bank, both the data bank which contains the memory address specified by the write request and any parity banks which utilizes that data bank will contain invalid data. The code status table keeps track of the locations of invalid data so the access scheduler does not erroneously serve read requests with stale data.

{\color{blue}
Figure~\ref{fig:coded_access_scheduler} depicts one implementation of the code status table. This is the implementation used to generate the simulation results described in sections 5 and 6. The implementation contains an entry for every row in each data bank. Each entry can take one of three values. The values indicate that either the data in both the data bank and parity banks is fresh, the data bank contains the most recent data, or one of the parity banks contains the most recent data. It is not necessary for the code status table to know which parity bank a write request was served, because the dynamic coding unit described later in this section keeps track of this information. We assume that the elements of the code status table are accessible at a very fast rate. 
}

{\color{red}
This implementation of the code status table can be improved. This code status table does not keep track of the intermediate steps the access scheduler takes when rebuilding codes after a write is served. When rebuilding the memory in two parity banks after a data bank has been written to, it is likely that elements of one parity bank will be restored before the other. The restored parity bank is ready to serve more memory requests using the rebuilt row, but the code status table will indicate that all the parity banks are unavailable until all parity banks are restored. Full knowledge of the status of all data and parity banks allows the access scheduler to serve more requests in some scenarios. \Matt{is this example necessary? Is the tangent worth the insight?}
}

%-----------------------
\begin{figure}[t!]
	%\includegraphics[width=0.9\linewidth]{fig/Read-algo.pdf}
	\includegraphics[width=0.96\linewidth]{fig/read_pattern_algo.png}
	\caption{{Description of the algorithm to build a read request pattern to be served in a given memory cycle.}}
	\label{fig:readAlgo}
\end{figure}
%-------------------------
%-----------------------
\begin{figure}[htbp]
	\centering
	%\includegraphics[width=0.7\linewidth]{fig/readAlgoAccessPattern.pdf}
	\includegraphics[width=0.96\linewidth]{fig/Read-Algo-Example.pdf}
	\caption{{Illustration of the algorithm to build a read request pattern to be served in a given memory cycle. All the read requests associated with the strikethrough elements are scheduled to be served in a given memory cycle. The figure also shows the elements downloaded from all the memory banks in order to serve these read requests.}}
	\label{fig:readAlgoAccessPattern}
\end{figure}
%------------------------
%\subsection{Read Algorithm for Coded Memory system}
\subsection{Read pattern builder}
\label{sec:readCodingAlgo}

{\color{blue}
A principal goal of the proposed memory system is to serve many read requests in a single memory cycle, and the redundant memory provided by the parity banks gives the memory controller the potential to fulfill this goal. The access scheduler must determine how to use the memory provided by the parity banks. When serving read requests, the access scheduler selects a set of requests to be scheduled from the bank queues. In order to select the set of requests to be scheduled, the access scheduler must determine how the memory requests will be served by the data and parity banks. Serving a memory request from a data bank is straightforward, because the symbols in the data banks uncoded, so they are ready to be used as long as the code status table indicates that the symbols are up-to-date. Serving a memory request using a parity bank more complex, because parity banks which contain coded symbols must use symbols downloaded from data banks in order to be decoded. The access scheduler uses the read pattern builder algorithm to determine which requests to serve using parity and data banks. \Matt{Is this paragraph too wordy?}
}

The read pattern builder selects which memory requests to serve and determines how requests served by parity banks will be decoded. The algorithm is designed to serve many read requests in a single memory cycle. Figure \ref{fig:readAlgo} is one possible implementation of the read pattern builder. It is important to note that the algorithm depicted will not always schedule the maximum number of read requests in a single memory cycle. We use the implementation shown here in our simulations described in sections 5 and 6. 

Figure ~\ref{fig:readAlgoAccessPattern} shows the algorithm depicted in one scenario. First, the read pattern builder marks $a(1)$ to be read from data bank $\mathbf{a}$. It then looks through banks $\mathbf{b}$, $\mathbf{c}$, and $\mathbf{d}$ searching for requests for rows $b(1)$, $c(1)$, or $d(1)$ because these symbols can be decoded from a parity bank using the $a(1)$ symbol. In this scenario $b(1$), $c(1)$, and $d(1)$ are all present in the bank queues and are served using parity banks. Symbols equal to  $a(1) + b(1)$, $a(1) + c(1)$, and $a(1) + d(1)$ are all downloaded from parity banks and decoded with $a(1)$. Next, $b(2$) is read from a data bank. Similar to before, $c(2)$ and $d(2)$ are served by downloading $b(2) + c(2)$ and $b(2) + d(2)$ symbols from the parity banks. Again as before, $c(3)$ is read from data bank and $d(3)$ is decoded using $c(3)$ and $c(3) + d(3)$. Finally, $d(4)$ is read from a data bank. In this scenario, Only the top loop of the read pattern builder as pictured in Figure~\ref{fig:readAlgo} schedules reads, but there are scenarios where the bottom loop is useful. \Matt{I have an optimal algorithm for scheduling read requests. Should I include it in this section?}

\begin{remark}
{\color{blue}
Here we note that the aforementioned approach of maximizing the number of read request being served per cycle does come with a cost. It increases the chances of having out-of-order execution of memory access requests. This does not pose a problem in the case when the memory requests go out of order for different cores. However, in order to prevent the out-of-order execution of the access requests arising from the same core, the logic needs to take care of in-order execution of requests from each cores. We assume that the code arbiter only admits requests into the bank queues if the requests can be immediately served without the risk of out-of-order execution.
}

\end{remark}
\ignore{
%-----------------------
\begin{figure}[htbp]
\centering
	\includegraphics[width=\linewidth]{fig/writealgo.pdf}
	\caption{{\bf Flowchart of Write Algorithm}}
	\label{fig:writeAlgo}
\end{figure}
%-------------------------
%\ignore{
\begin{itemize}
\item Write about how we solve the out of order look ahead problem. If we solve 
	it at all ????
\end{itemize}
}
%\subsection{Write Algorithm for Coded Memory system}
\subsection{Write pattern builder}
\label{sec:writeCodingAlgo}
The inclusion of parity banks allows the memory controller to serve additional write requests per cycle. The memory controller can serve multiple writes which target a single data bank by committing some of the writes to parity banks. Similar to the read pattern builder, the access scheduler implements a write pattern builder algorithm which determines which write requests to schedule in a single memory cycle. 

%-----------------------
\begin{figure}[htbp]
\centering
	\includegraphics[width=\linewidth]{fig/write_pattern_algo.png}
	\caption{{ Flowchart of write pattern builder}}
	\label{fig:writeFlow}
\end{figure}
%-------------------------

Figure~\ref{fig:writeFlow} illustrates a potential implementation of the write pattern builder. The implementation of the write pattern builder discussed here is used in the simulations described in sections 5 and 6. Only when the write bank queues are nearly full does the access scheduler execute the write pattern builder algorithm. 

Figure~\ref{fig:writeAlgoAccessPattern} shows how the write pattern builder described in Figure~\ref{fig:writeFlow} performs in one scenario. Without the parity banks only one write request can be scheduled for each of the four data banks. The inclusion of parity banks allows for 10 write requests to be scheduled. Note that an element which is addressed to row $n$ in a data bank can only be written to the corresponding row $n$ in the parity banks. In this scenario, the write queues for each data bank are full. The controller takes $2$ write requests from each queue and schedules one to be written to their target data bank the other to a parity bank. The controller also updates the code status table. \Matt{The figure figure here contains an error - 10 write requests should be served}

{\color{red}
Figure~\ref{fig:writeAlgoAccessPattern} also demonstrates how the code status table changes to reflect the freshness of the elements in the data and parity banks. Here, the 00 status indicates that all elements are updated. The 01 status indicates that the data banks contain fresh elements and the elements in the parity banks must be recoded. The 10 status indicates that the parity banks contain fresh elements, and that the data bank must be updated and the elements in the parity banks must be updated.
}

%-----------------------
\begin{figure}[t!]
\centering
         \includegraphics[width=\linewidth]{fig/Write-Algo-Example.pdf}
	\caption{Figure describing write algorithm access pattern}
	\label{fig:writeAlgoAccessPattern}
\end{figure}
%-------------------------
\subsection{ReCoding unit}
\label{sec:recoding}
After a write request has been served, the stale data in the parity or data banks must be replaced. The ReCoding Unit is responsible for updating the elements of data and parity banks after a write is served. The ReCoding Unit contains a queue of {\em recoding requests}. Every time a write is served, recoding requests are pushed on to the queue. Recoding requests indicate which data and parity banks contain stale elements, and they indicate the bank the write was served to which generated the recoding request. The recoding requests also contain the cycle number the request was created so the ReCoding Unit may prioritize older requests. 

\subsection{Dynamic Coding}
\label{sec:dynamicCoding}
To reduce memory overhead, the size of the parity banks is designed to be only a fraction of the size of the data banks. Ideally, the most heavily accessed portions of memory are stored in the parity banks. The dynamic coding block is responsible for maintaining codes for the most heavily accessed memory sub regions.

\subsubsection{Motivation}
Bank conflicts are most likely to occur when regions of shared-memory are localized to certain memory regions. Multi-core systems often generate memory access requests to overlapping memory regions. By dynamically coding certain memory locales, the proposed memory system aim to resolve the bank conflicts which occur during periods of heavy memory access in multi-core systems.

\begin{figure}[htbp]
		\includegraphics[width=\linewidth]{fig/dedup_whole.png}
		\caption{Memory Access from the Dedup PARSEC benchmark. This trace was generated using 8 cores.}
		\label{fig:dedup_whole}
\end{figure}

\begin{figure}[htbp]
		\includegraphics[width=\linewidth]{fig/dedup_dense.png}
		\caption{Memory Access from the Dedup PARSEC benchmark demonstrating the density of memory accesses}
		\label{fig:dedup_dense}
\end{figure}

An examination of the memory trace from one of the PARSEC benchmarks illustrates a scenario where dynamic coding works well. Figure~\ref{fig:dedup_whole} shows the memory trace of a simulation of an 8-core system running the dedup PARSEC benchmark. The y-axis shows the address accessed by the cores. The x-axis shows the access time in nanoseconds. This plot shows that most of the accesses from various cores are primarily located in the lower memory band. Greater than 95\% of all memory accesses are in this band. Figure~\ref{fig:dedup_dense} magnifies this band and reveals that the lower band is composed of two sub-bands of roughly equal density. In a scenario where the dynamic coder can choose to encode two memory blocks it would  would detect that nearly all memory access are localized to the primary memory bands, so only those regions would be encoded.

\subsubsection{Encoder Design}
There are many possible implementations of the dynamic coding unit. The design described here is used in the simulator used to generate the results described in sections 5 and 6.

The {\em dynamic coding} block splits the each memory bank according to the memory partition coefficient {\em r}. Each bank is split into $\lceil\frac{1}{r}\rceil$ partitions. Recall that $\alpha$ is the maximum memory overhead of the proposed memory system. The block can select up to $\frac{\alpha}{r} - 1$ regions to be encoded in the parity banks. A single region is reserved to allow the dynamic coding block to encode a new region.

Every $T$ ticks, the {\em dynamic coding} unit chooses the $\frac{\alpha}{r} - 1$ regions with the greatest number of memory accesses. The dynamic coding unit will then encode these regions in the parity banks. If all the selected regions are already encoded, the unit does nothing. Otherwise, the unit begins encoding the most accessed region. Once the dynamic coding unit is finished encoding a new region, the region becomes available for use by the rest of the memory controller. If the memory ceiling $\alpha - r$ is reached when a new memory region is encoded, the unit evicts the least frequently used encoded region.

{\color{blue}
\subsection{Prefetching Codes}
\label{sec:prefetching}
Dynamic coding works best when the most heavily accessed regions of memory do not change over time. Though dynamic coding can still be effective when the memory access trend is not static, the proposed memory system can benefit from a system which anticipates sequential memory accesses. 
The prefetcher attempts to detect sequential memory accesses and exploit idle memory banks to potentially server future memory requests. The prefetcher analyzes the pattern of memory accesses over a fixed number of memory cycles and detects sequential memory accesses. The prefetcher prioritizes long sequential memory accesses as motivation for performing an anticipatory read. Because of the speculative nature of the prefetcher, it is given the lowest priority of all the components in the access scheduler, and it will only schedule a memory access to a memory bank if all the other units do not do so first. 
} \Ethan{More details here}


%%%%%%%%%%%%%%%%%%%%%%%%%%%%%%%%%%%%%%%%%%%%%%%%%%%%%%%%
% Experimental Methodology
%%%%%%%%%%%%%%%%%%%%%%%%%%%%%%%%%%%%%%%%%%%%
\section{Experimental Methodology}
\label{sec:experimentalmethodology}

In this section, we discuss our method for evaluating the performance of the proposed memory systems. We utilize the PARSEC v2.1 and v3.0 benchmark suites with the gem5 simulator to generate memory traces. Next, we run the Ramulator DRAM simulator to to measure the performance of the proposed memory systems. We compare the baseline performance of the Ramulator simulators against a modified version of the Ramulator simulator which implements the proposed memory systems.

\subsection{Memory Trace Generation}

We use the PARSEC benchmark suite to evaluate the performance of the proposed memory systems. The PARSEC benchmark suite was developed for Chip Multiprocessors and is composed of a diverse set of multithreaded applications~\cite{bienia09parsec2}. The benchmarks allow us to observe how the proposed memory systems perform in dense memory access scenarios. A number of input sets are provided alongside the PARSEC benchmarks. To run the PARSEC applications, we use the gem5 simulator~\cite{parsec_2_1_m5}.

The gem5 simulator allows us to select the number of processors and their attributes we use to generate the memory traces. For most traces, we used 8 processors for the PARSEC benchmarks we evaluated. We also used 16 and 32 processors to explore the effects of denser memory traces on the Ramulator simulation results. The PARSEC applications can be divided into multiple regions where the nature of the computation therein differs. The most computationally interesting region is the region where parallel processing takes place. We extract the region of the trace where the application was performing parallel processing, as it is this region where there is a high probability for bank conflicts to occur. Thus, our Ramulator simulations are run only on this parallel portion of the PARSEC benchmarks.

\Matt{TODO: List the input sets used for each benchmark(?)}

We convert the gem5 memory traces to the Ramulator CPU-trace format. The conversion process is simple as it only requires the reorganization of the memory addresses in the gem5 memory trace. It is important that we use the Ramulator CPU-trace format and not the DRAM-trace format because the CPU-trace format contains information necessary to simulate the processor subsystem within Ramulator, the subsystem wherein the memory controller lies.

\subsection{PARSEC Trace Attributes}

The most important attributes of the memory traces as it relates to the proposed memory systems is the density of the traces, the overlap of the memory accesses between the processors, and how stationary the heavily utilized regions of memory are. The PARSEC benchmarks are sufficiently dense as illustrated by Figure~\ref{fig:dedup_dense}. It is clear from this image that there is heavily memory utilization during this section of the Dedup benchmark. On average across all processors, there is an average of 1.11 nanoseconds between memory accesses per core. The equates to an average of 2.22 cycles between memory access requests per 2 Ghz processor. 

The location of the most heavily used memory region is stationary with respect to time for all PARSEC benchmarks. Figure~\ref{fig:dedup_whole} shows the whole of a dedup memory trace. There are two major bands clear from this image, and the bands remain horizontal for the entirety of the plot indicating that these bands remain heavily accesses for duration of the trace. Figure~\ref{fig:dedup_dense} is a magnified view of the bottom band. This figure reveals that the bottom band is composed of two sub-bands which are also stationary with respect to time. The structure of the dedup the memory trace is representative for all the PARSEC benchmarks. It is also clear from this image that the memory regions utilized by all of the processors overlap sufficiently to create bank conflicts.

\subsection{Ramulator}

We use the Ramulator DRAM simulator to compare the number of CPU cycles required to execute the PARSEC memory traces. We use the vanilla Ramulator simulator to acquire the baseline number of CPU cycles. We extended the memory controller in Ramulator in order to simulate the proposed memory system, and we use the modified Ramulator to examine the improvements the memory system has over the baseline. We use a consistent Ramulator configuration file so that the improvements we observe over baseline are purely a result of the memory system resolving bank conflicts. We test across the amount of memory overhead $\alpha$ the memory system is permitted to use. 
The following are the specifics of the Ramulator configuration file used to acquire the simulation results:
\begin{itemize}
\item Standard : HBM
\item Channels: 8
\item Ranks: 1
\item Speed: 1 Gigabits per second
\item Organization: 4 Gigabits
\item CPU ticks: 32
\item Memory ticks: 5
\end{itemize}



%%%%%%%%%%%%%%%%%%%%%%%%%%%%%%%%%%%%%%%%%%%%%%%%%%%%%%%%
% Results
%%%%%%%%%%%%%%%%%%%%%%%%%%%%%%%%%%%%%%%%%%%%
\section{Simulation Results}
\label{sec:simulation}
The simulation results are consistant across all the PARSEC benchmarks. Given sufficient memory overhead, we see that we consistantly achieve a roughly 25\% improvement over the baseline simulation. We generally find that the Coding Scheme I performs the best out of the proposed schemes. 

\subsection{PARSEC Results}


\subsection{PARSEC augmentation}
Because the PARSEC benchmarks are homogenous in structure, we chose to augment them in order to observe how the proposed memory system performs in more scenarios. The PARSEC benchmarks were augmented in two ways. The first augmentation is to split the memory bands observed in Figure~\ref{fig:dedup_whole} and Figure~\ref{fig:dedup_dense} into a greater number of bands. The second augmentation was to ramp the memory bands over time.	


\subsection{Augmented PARSEC Results}

\subsection{Design Parameters}
\Matt{This section is old and needs to be updated to reflect our shift in focus away from LTE/UMTS}
In this section, we discuss various parameters that we consider 
%in this phase of the project 
to design and simulate the efficient code storage in this project. \\
\textit{Memory overhead}: 
%The gains of multiple accesses to a memory bank every cycle comes with an 
%associated cost. 
The crucial cost in coded memory system is to store the compressed redundancy or 
the codes. The extra memory space used to store these codes should limit to 
15$\%$ of the overall memory.\\
\textit{Memory size}: The memory size and the parity storage size decide the 
code function to be used to essentially compress the redundant data. This design 
considers a portion of memory to be coded. \\ \textit{Memory Banks}: The memory 
banks essentially are the units which store the data. We consider the code 
design for 8 memory banks. We consider the memory banks addressed with Least 
Significant Bits (LSBs) of the address. The last 3 bits of the address decide 
which bank, the memory address belongs to and the rest of the MSBs decide the 
row location within the bank.\\ \textit{Cache line size}: The memory accesses 
are bundled in a burst as a cache line is evicted and is requested to be 
replaced by the cores. The cache controller thus requests a cache line which is 
a starting address and the length of the cache line. In this design, we consider 
cache line size of 128 bytes and 256 bytes. However, each core can potentially 
have a different cache line size and the concept of coding could be extended for 
various sizes.\\ \textit{Element size}:  Each memory location in a memory bank 
stores 256 bit of data. This essentially relates to decoding/understanding the 
address access request to the memory bank. The cores request memories to be read 
or written for multiple elements. For example, a core with 128 bytes of cache 
line would request 4 elements of read/write for each cache line. The shared 
traces have two different request patterns, for 128 bytes and for 256 bytes. \\
\textit{Number of Cores}: This parameter refers number of cores making access to 
the memory controller. This parameter is not used in the design of the coding 
scheme. However, we validate the design using the 6 core access trace shared 
with us for LTE and UMTS. \\
\textit{Access rate}: This is the average rate at which the memory controller 
executes the reads/writes to the memory banks. In this design, we consider 1.54 
ns as the access rate. This would mean that the clock rate to memory would be at 
650MHz. This parameter is required to simulate the performance for the shared 
traces.  


%%%%%%%%%%%%%%%%%%%%%%%%%%%%%%%%%%%%%%%%%%%%%%%%%%%%%%%%
% Conclusion
%%%%%%%%%%%%%%%%%%%%%%%%%%%%%%%%%%%%%%%%%%%%


%%%%%%%%%%%%%%%%%%%%%%%%%%%%%%%%%%%%%%%%%%%%%%
% Acknowledgements
%%%%%%%%%%%%%%%%%%%%%%%%%%

\section{Acknowledgements}
This document is derived from previous conferences, in particular HPCA 2017.  We thank Daniel A. Jimenez,  Elvira Teran for their inputs.



%{\color{red}\textbf{Coding over small number of banks:~}In order to save storage space allocated to different pointers.....This is something which does not arise in many classical scenarios where coding is employed. For example, in communications, it is preferable to encode over long messages as apart from computational complexity as in most cases there is no additional penalty (proportional to the code length) in terms of pointers required to deal with write requests.}

%In this part, we attempt to design efficient codes based on the memory traces 
%shared by Huawei.  The goal of this design was to simulate the efficiency of 
%coding and compare the results to the baseline implementation of not coding.  
%During this design phase, we explored various code functions that could be used 
%to create the codes on the stored data. We decide upon using the XOR function to 
%store the data in the parity banks because of its low complexity overhead and 
%for preserving the linearity of codes. Linear codes offer the widest range of 
%functionality because any order of the codes may be used to either encode or 
%decode. This lack of dependency allows our design to use the parity banks in the 
%most flexible way possible. We also explore the potential benefits of using 
%different weights to the memory elements for the XOR function. For examples, the 
%memory elements $a_0$ and $b_0$ could be stored as $\alpha a_0 + \beta b_0$ for 
%integer values of $\alpha$ and $\beta$ which belong to any Galois Field. The 
%least complex design for the decoder would be for taking $\alpha$ = 1 and 
%$\beta$ = 1 . Another design consideration explored is the compression factor to 
%generate the codes.  The codes can be generated by using xor on 2 or more memory 
%elements. For example, suppose there are four banks A, B , C and D. Each of the 
%banks hold $a_0$ to $a_n$, $b_0$ to $b_n$ , $c_0$ to $c_n$ and $d_0$ to $d_n$ 
%elements respectively. The possible codes for these memories could be:
%\begin{equation}
%a_i + b_i, b_i + c_i, c_i + d_i \text{ and } c_i + a_i  \text{ for i = 0 to n }
%\end{equation}
%This scheme uses the combination of 2 memory elements to generate the codes.  
%Although this requires 100$\%$ extra memory overhead, it enables 100$\%$ extra 
%memory accesses per cycle, i.e., 4 extra accesses in this case.
%Another design could be to compress the codes by combining all 4 memory elements 
%to generate the codes:
%\begin{equation}
%a_i + b_i + c_i + d_i \text{ for i = 0 to n }
%\end{equation}
%This design gives one extra access per cycle at the cost of 25$\%$ memory 
%overhead. However, the decoder here needs to know 3 elements to be able to 
%decode the 4th element. So although we are able to compress more data into a 
%single memory location, it comes with the cost of additional memory logic.  The 
%scheme described above "codes" the memory banks using elements from different 
%banks. We call this type of coding as Interbank Coding. We also explore the 
%orthogonal way of coding, i.e. intra-bank coding where we use the memory 
%elements from the same bank to generate codes.
%\\
Following are the objectives used in code design:
\begin{itemize}
\item Read access : 4 per bank in one cycle
\item Write access : 2 per bank in one cycle
\item Shared Memory size 8 kB - 256 kB
\item Number of Banks : 8
\item Memory overhead : 15$ \% $
\item Parity banks : 5 or 6 shallow banks for code storage
\end{itemize}


%%%%%%% -- PAPER CONTENT ENDS -- %%%%%%%%

\newpage

%%%%%%%%% -- BIB STYLE AND FILE -- %%%%%%%%
%\bibliographystyle{ieeetr}
\bibliographystyle{IEEEtran}
\bibliography{references}
%%%%%%%%%%%%%%%%%%%%%%%%%%%%%%%%%%%%

\end{document}
