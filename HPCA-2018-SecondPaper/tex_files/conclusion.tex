\section{Conclusion}
\label{sec:conclusion}
%\Ethan{Duplicate of abstract?}
This paper proposes a coded memory system that is amenable to multi-bank DRAM architectures such as HBM. Our Reed-Solomon based coding scheme reduces bank conflicts by creating many redundant ways to access a requested data element. This is achieved by assigning some banks to be dedicated parity banks. A memory controller then decides which frequently-used data elements should be stored in these parity banks. Simulations on the PARSEC benchmarks show significant reduction in CPU latency when the controller's reorder buffer is of sufficient length. Our detailed experimental results demonstrate the utility of our memory controller design for improving the performance of uncoded memory systems.

The parity banks in our proposed memory system improve the system's performance in several scenarios. The inclusion of parity banks brings additional memory overhead to the system. One potential method for minimizing this memory overhead is dynamic coding techniques which produce {\em shallow} parity banks. In dynamic coding, heavily accessed regions of memory are identified and selectively encoded into parity banks. Thus, small {\em shallow} parity banks are generated which can serve requests for the most heavily accessed regions of memory, and the inclusion of these shallow banks incur only a small memory overhead.

%In this paper, we explore the use of coding theory in DDR and high bandwidth memory (HBM) system. Conventional DDR systems use timing optimization techniques around DDR protocol to improve the efficiency of accesses. In this work we propose increasing redundancy to help distribute access across DRAM pages. This will help in efficient retrieval of data. These schemes will be evaluated and optimized with several iterations. The aim is to find the best performing code design scheme given the performance requirement compared to alternatives. Our proposed scheme results in a decrease in the number of contended memory accesses between cores, therefore reducing the overall latency of the system. The reduction in the latency can be seen directly as an increase in the overall system performance. We show that with a memory overhead of  $15$\%, we can enable $4$ extra accesses to a bank while remaining within the given design parameters. We also included emphasis on the code design and optimization of the access scheduler and made significant process in improving the design emulation on the Ramulator platform.
%
%This paper explores the use of coding theory in double data rate (DDR) and high bandwidth memory (HBM) systems. Conventional DDR systems use timing optimization techniques around the DDR protocol to improve access efficiency. This paper proposes utilization of coding techniques to store data across DRAM pages in a redundant manner. The introduced redundancy creates multiple alternatives to access a particular data element stored in the DRAM; consequently making the data retrieval more efficient. This paper proposes a memory design which is based on Reed Solomon codes. The proposed design decreases the number of contended memory accesses between cores. This helps reduce the overall latency of the system, leading to increased performance. Various benchmarking experiments carried out on the proposed design show that the designed memories can enable $4$ extra accesses to a bank while incurring a storage overhead of only $15$\% and remaining within the acceptable values of other design parameters. 

%{\color{blue} \noindent \textbf{Potential improvement using prefetching:}~
%\Ethan{Remove entire subsection?}
%\noindent \textbf{Future Work--Prefetching:}~
\Matt{Is prefetching worth bringing up? Shouldn't this be somewhere before the conclusion?} One potential source of improvement is the addition of a memory prefetching unit, similar to an instruction or cache prefetching unit, which can detect linear access patterns to memory regions. For example, if a sequence of memory accesses is issued in increasing order spaced one byte apart, then a prefetching unit would predict the next access to be one byte past the previous one. 
Prefetching-based memory designs have been studied only in the context of {\em uncoded} memory systems~\cite{Kim2016, Kadjo2014, Shevgoor2015, JL2013}. We can augment our scheme by fetching a predicted address from a parity bank during accesses for which it remains valid but idle. Then for future memory accesses, the controller can check the prefetched data and attempt to complete the request using current accesses and prefetched data. This means that previously occupied banks are available to serve more accesses per cycle. As memory accesses wait to be issued in the bank queues, they can simultaneously be checked with prefetched data. Thus, no extra latency is anticipated by the addition of a memory prefetching unit.
%
Figure~\ref{fig:prefetch1} shows two plots of memory accesses for several banks across time. Both figures suggest linear access patterns and thus larger performance improvements when coding caching is combined with prefetching for this application.

%
%To the best of our knowledge, the application of prefetching schemes with coded memory systems has not been explored before.
%
%For example, say the prefetcher sees 2 consecutive memory requests in a row. It then predicts that the next two accesses, locations $a_0$ and $b_0$, are likely to be accessed in the near future. It reads $a_0 + b_0$ from the parity bank for future use. Next, access to location $a_0$ and $b_0$ are issued to the memory. Now, instead of reading both $a_0$ and $b_0$, only a single location has to be read from in memory, while the other location can be obtained from the prefetched data. This allows for an additional access to be issued from the now free memory bank.  In these cases, it is possible to obtain up to two additional memory accesses in a given cycle, one from the prefetched data and one from the parity bank.
%
%Implementation of a memory prefetch should only require overhead for space and the associated logic to implement it. Since memory accesses are often stalled due to bank conflicts, checking pending accesses to the prefetched data should require no additional time overhead. 
%
%\noindent {\bf Coding and prefetching:~}One of the ideas explored in this paper is that of proactive prefetching of data from (unused) memory banks to buffers based on the pattern of pending access requests at the memory controller. This creates the opportunities to serve multiple access requests in a given memory cycles and also tries to utilize all the available memory banks throughout the operation of the memory system. Such prefetching based memory designs in the context of {\em uncoded} memory systems has been previously studied in the literature (see e.g.,~\cite{Kim2016, Kadjo2014, Shevgoor2015, JL2013}). To the best of our knowledge, the application of prefetching schemes with coded memory systems has not been explored before.